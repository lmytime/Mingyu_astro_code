\documentclass[CJK, utf8, GBK, oneside, a4paper, 12pt]{ctexart}
\usepackage{inputenc}
\usepackage{geometry,enumerate,amsmath}
\usepackage{float}
\usepackage{graphicx}
\usepackage{multirow}
\usepackage{fancybox}
\usepackage{bookmark,amssymb,indentfirst,bm,subfigure,hyperref,pifont}
\usepackage[table,xcdraw]{xcolor}
\usepackage[normalem]{ulem}
\usepackage{wrapfig}
\usepackage[font=small]{caption}
\usepackage{ listings} 
\lstset{
    %backgroundcolor=\color{red!50!green!50!blue!50},%代码块背景色为浅灰色
    rulesepcolor= \color{gray}, %代码块边框颜色
    breaklines=true,  %代码过长则换行
    numbers=left, %行号在左侧显示
    numberstyle= \small,%行号字体
    %keywordstyle= \color{blue},%关键字颜色
    commentstyle=\color{gray}, %注释颜色
    frame=shadowbox%用方框框住代码块
    }

%页边距
\geometry{left=2.0cm,right=2.0cm,top=2.0cm,bottom=2.0cm}

%页眉页脚
\usepackage{fancyhdr}
\pagestyle{fancy}
\lhead{\kaishu 学习笔记} 
\chead{} 
\rhead{\thepage} 
\lfoot{} 
\cfoot{\thepage}
\rfoot{} 
\renewcommand{\headrulewidth}{0.4pt} 
\renewcommand{\footrulewidth}{0.4pt}


\useunder{\uline}{\ul}{}


%修改章节号
%\renewcommand\thesection{\Alph{section}}
%\renewcommand\thesection{\Alph{subsection}}

%新的引用
\newcommand{\upcite}[1]{\textsuperscript{\textsuperscript{\cite{#1}}}}


\newcommand{\myfiguresetwidthfilecaplabel}[4]{
    \centering
    \includegraphics[width = #1\linewidth]{./images/#2}
    \captionsetup{font={footnotesize}}
    \caption{#3}
    \label{#4} }

%摘要和关键词
\renewenvironment*{abstract}[1]{%
\newcommand\gjc{#1}
\paragraph{摘要:}
}{\paragraph{关键词:}\gjc }

%求导的d
\newcommand{\rd}{\mathrm{d}}

%让公式变大
\newcommand{\dps}[1]{\displaystyle{#1}}

\begin{document}

%插入封面
\thispagestyle{empty}
 \begin{figure}[H]
        \centering
        \includegraphics[scale=1.2]{cover_0/Tsinghua.png}
 \end{figure}
\vskip0.5cm
\begin{center}
    \makebox[109mm][s]{\heiti\zihao{-0}\bf 学习笔记}
\end{center}
\vskip1.5cm
\begin{center}
    \makebox[20mm][s]{\heiti\zihao{4} 主题}\underline{\makebox[110mm][c]{\heiti\zihao{3}天文}}\\
    \vskip0.8cm
    
    \makebox[20mm][s]{\heiti\zihao{4} 专业}\underline{\makebox[110mm][c]{\heiti \zihao{3} 物理学}}\\
    \vskip0.8cm
    
    \makebox[20mm][s]{\heiti\zihao{4} 姓名}\underline{\makebox[110mm][c]{\heiti \zihao{3} 李明宇}}\\
    \vskip0.8cm
    
    \makebox[20mm][s]{\heiti\zihao{4} 入学年份}\underline{\makebox[110mm][c]{\heiti \zihao{3} 2018年}}\\
    \vskip0.8cm
    
    \makebox[20mm][s]{\heiti\zihao{4} 指导教师}\underline{\makebox[110mm][c]{\heiti\zihao{3} 蔡峥}}\\
    \vskip0.8cm
    
    \makebox[20mm][s]{\heiti\zihao{4} 创建时间}\underline{\makebox[110mm][c]{\heiti\zihao{3} 二〇二〇年}}
\end{center}

\newpage
\thispagestyle{empty}
\setcounter{page}{1}
\newpage


\renewcommand\appendix{\setcounter{secnumdepth}{-2}}


%让公式变粗
\mathversion{bold}

%标题
\begin{center}
    \heiti\zihao{1}笔记\\
    \kaishu\zihao{3}Mingyu \ Li \qquad 李明宇 \qquad THU\  物理8
\end{center}

%摘要和关键词
\begin{abstract}{天文,学习笔记}
    记录一些学习笔记
\end{abstract}
\tableofcontents
\newpage
\begin{comment}
\begin{equation}\dps{ v_g=\frac{\rd \omega}{\rd k}=\frac{c}{n+\omega\frac{\rd n}{\rd \omega}}}\end{equation}

%插入图片示例
\begin{figure}[!htbp]
    \myfiguresetwidthfilecaplabel{0.92}{1}{周期性阻抗同轴电缆示意图}{F1}
\end{figure}

\end{comment}
\section{Photometry}

几乎所有我们能接受到的来自太阳系外的信息都是以电磁辐射(EMR)的形式。






\begin{thebibliography}{1}
\bibitem{OU} An Introduction to Astronomical Photometry Using CCDs, W. Romanishin, University of Oklahoma
\end{thebibliography}


\end{document}